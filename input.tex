\begin{document}
**Question 5**
**5.1** The turning point of f is \((-2, -5)\).

**5.2** To find the x-coordinates of A and B, we need to solve the system of equations:
$$f(x) = g(x)$$
$$-(x 3)^2   4 = x 5$$
Expanding and simplifying, we get:
$$-x^2 - 6x - 5 = x   5$$
$$-x^2 - 7x - 10 = 0$$
Using the quadratic formula, we find that:
$$x = \frac{-(-7) \pm \sqrt{(-7)^2 - 4(-1)(-10)}}{2(-1)}$$
$$x = \frac{7 \pm \sqrt{49   40}}{-2}$$
$$x = \frac{7 \pm \sqrt{89}}{-2}$$
Therefore, the x-coordinates of A and B are:
$$x = \frac{7   \sqrt{89}}{-2} \approx -2$$
and
$$x = \frac{7 - \sqrt{89}}{-2} \approx -5$$

**5.3** Hence, the x-coordinates of A and B are \( -5 \) and \( -2 \), respectively.

**5.4** To determine the values of \(c\) for which the equation \((x-c 3)^2   4 = x c 5\) has one negative and one positive root, we need to find the discriminant of the quadratic equation:
$$b^2 - 4ac$$
Substituting \(a = 1\), \(b = -c 7\), and \(c = c 5\), we get:
$$(-c 7)^2 - 4(1)(c 5)$$
$$c^2 - 14c   49 - 4c - 20$$
$$c^2 - 18c   29$$
For the equation to have one negative and one positive root, the discriminant must be positive:
$$c^2 - 18c   29 > 0$$
We can solve this inequality using the quadratic formula:
$$c = \frac{-b \pm \sqrt{b^2 - 4ac}}{2a}$$
Substituting \(a = 1\), \(b = -18\), and \(c = 29\), we get:
$$c = \frac{-(-18) \pm \sqrt{(-18)^2 - 4(1)(29)}}{2(1)}$$
$$c = \frac{18 \pm \sqrt{324 - 116}}{2}$$
$$c = \frac{18 \pm \sqrt{208}}{2}$$
$$c = \frac{18 \pm 2\sqrt{52}}{2}$$
$$c = 9 \pm \sqrt{52}$$
Therefore, the values of \(c\) for which the equation \((x-c 3)^2   4 = x c 5\) has one negative and one positive root are \(c = 9   \sqrt{52}\) and \(c = 9 - \sqrt{52}\).

**5.5** To find the maximum distance between \(f\) and \(g\) in the interval \(x_1 \le x \le x_2\), we need to find the maximum value of the function \(|f(x) - g(x)|\) in that interval.
$$|f(x) - g(x)| = |-(x 3)^2   4 - (x 5)|$$
$$= |x^2   6x   9   4 - x - 5|$$
$$= |x^2   5x   8|$$
Since the function \(|x^2   5x   8|\) is a parabola that opens upwards, its minimum value occurs at its vertex. The vertex of the parabola is at \(x = -\frac{b}{2a} = -\frac{5}{2(1)} = -\frac{5}{2}\). The minimum value of the function is therefore:
$$|f(-\frac{5}{2}) - g(-\frac{5}{2})| = |(-\frac{5}{2})^2   5(-\frac{5}{2})   8 - (-\frac{5}{2}   5)|$$
$$= |\frac{25}{4} - \frac{25}{2}   8 - \frac{3}{2}|$$
$$= |\frac{25}{4} - \frac{50}{4}   \frac{32}{4} - \frac{6}{4}|$$
$$= |\frac{5}{4}|$$
$$= \frac{5}{4}$$
Therefore, the maximum distance between \(f\) and \(g\) in the interval \(x_1 \le x \le x_2\) is \(\frac{5}{4}\).

**5.6** If \(h(x) = f(x)   k\), then the graph of \(h(x)\) is the graph of \(f(x)\) shifted up by \(k\) units. Therefore, the graph of \(h(x)\) will pass through the point \((-2, -5   k)\). Since the graph of \(h(x)\) also passes through the point \((0, 0)\), we have:
$$h(0) = 0$$
$$f(0)   k = 0$$
$$-5   k = 0$$
$$k = 5$$
Therefore, the equation of \(h(x)\) is:
$$h(x) = f(x)   5 = -(x 3)^2   4   5$$
$$h(x) = -x^2 - 6x - 1   9$$
$$h(x) = -x^2 - 6x   8$$
\end{document}
